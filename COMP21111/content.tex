% Set the author and title of the compiled pdf
\hypersetup{
	pdftitle = {\Title},
	pdfauthor = {\Author}
}

\section{Setting the stage}

Before it is possible to understand some of the content in these notes, it is
required to have a basic knowledge of other content that was not covered in my
previous notes. This section attempts to give an overview of the relevant
topics.

\subsection{Decision problems}

A decision problem is essentially a question, described by some formal language
(such as that of propositional logic) with a yes/no answer. In order to obtain
whether the decision problem yields yes or no, input parameters must be
specified and the problem evaluated according to the language it is written in.

Another requisite of a decision problem is that the possible set of inputs must
be infinite. That is to say that $5 = 5$, or ``Is the sky blue?'' would not be a
decision problem, since the set of inputs in both cases is the $\emptyset$.
Likewise, if a problem doesn't yield a yes/no answer (for example, a quadratic
equation), then it is not a decision problem.

\subsubsection{Decidable problems}

A problem is classed as decidable if it is both a decision problem and there
exists and algorithm that will be able to compute the correct answer for all
(infinite number of) inputs to the problem. Such an algorithm can be described
by any Turing complete programming language, in fact, it works both ways; if
there is a computer program which can find the correct yes/no answer for a
decision problem over all inputs, then the problem must be decidable.

Some decision problems are undecidable. It is impossible to create an algorithm
that can always solve the problem for all of its inputs. One such problem is the
Halting problem, which asks:

\begin{quotation}
  Given the description of an arbitrary program and a finite input, decide
  whether the program finishes running or will run forever.
\end{quotation}

It has been proven that the Halting problem is undecidable when running on a
Turing machine. The essence of the proof is that the algorithm evaluating
whether the input program will halt or not could be made to contradict itself.

\subsection{Mappings}

A mapping is a function that takes an input from one set and returns an output
from another (or the same) set. Conversely, you could describe a function as
mapping one set to another.

The notation $ f'(x) = f(x) + \{ a \rightarrow b\}$ means:

\[
  f'(x) \definedas
  \begin{cases}
    b    & \text{if x = a}\\
    f(x) & otherwise
  \end{cases}
\]

\subsection{Binary relations}

Binary relations are often denoted by $\Rightarrow$, and it's reverse is denoted
by $\Leftarrow$ such that $y \Leftarrow x \definedas x \Rightarrow y$. Along a
similar train of thought, the symmetric closure of $\Rightarrow$ is
$\Leftrightarrow$, where $\Leftrightarrow \definedas \Rightarrow \cup
\Leftarrow$

\subsection{Orders}

An order is a relation that is irreflexive and transitive. This means that each
pair in the relation cannot be constructed of only one element (e.g. $x > x$
wouldn't be allowed) and if $((x > y) \wedge (y > z)) \implies x > z$.

\subsection{Directed graphs}

A directed graph consists of a set $N$ and a binary relation on the set $R$. The
elements in $N$ are nodes, and the relation $R$ defines the edges between the
nodes. A directed graph is finite if it's set is finite.

A \textit{path} is a subset of $R$ where each element of the path will end at
the start of another element with the exception of the start and end elements. A
cycle is a path where there is no start and end pairs.

\subsubsection{Directed Acyclic Graphs (DAG's)}

A Directed Acylcic Graph is a directed graph that has no cycles. If a dag has a
node $n$ such that every other node in the dag is reachable by $n$, then the dag
is \textit{rooted at $n$}.

\subsection{Directed acylcic graphs}



\section{Propositional logic}

\section{Evaluating formulae}

\section{Satisfiability}

\section{CNF}

\section{Definitional Clausal Form Translation}

\section{DPLL}

\section{Encoding problems in SAT}

\section{Randomly generated clause sets}

\section{Randomised algorithms for satisfiability checking}

% Lecture 10
\section{Signed Formulae}

A signed formula is one where there is an expression, $A$ and a boolean $b$,
where $A = b$. If $A = b$ is {\it true}, in an interpretation $I$, then it is
denoted by $I \models A = b$, and consequently, $I$ is a model of the signed
formula $A = b$.

\marginpar{This is also when $I(A) =b$.}

If a signed formula has a model, then it is specifiable.

\subsection{Finding a model of a specifiable formula}

If we had a signed formula such as $A \Leftrightarrow B = 1$, the three
interpretations that model it are:

\begin{center}
    \begin{tabular}{cc}
        A & B\\ \hline
        0 & 0\\
        0 & 1\\
        1 & 1\\
    \end{tabular}
\end{center}

\subsection{Tabelau}

A tableau is a tree with each node being a signed formula. The tableau for the
signed formula $A = b$ would have the root node as $A = b$.

The notation for a set of branches is $B_1 | ... | B_n$, where each $B_i$ is a
branch.

\subsubsection{Branch expansions}

There are a number of rules that can be used to expand the branches of a
tableau.
