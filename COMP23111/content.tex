% Set the author and title of the compiled pdf
\hypersetup{
  pdftitle = {\Title},
  pdfauthor = {\Author}
}

\section{An introduction to Database Management Systems}

DataBase Management Systems (DBMS's) are a type of middleware that provide a
layer of abstraction for dealing with databases. It is nearly always unnecessary
to write software from scratch that interfaces with a database, since a lot of
database operations will share a significant amount of logic.

Henceforth, a lot of the functionality required of applications that make use of
a database is placed into a DBMS, which application developers can make use and
save time. The DBMS acts as a service, that is well implemented and is able to
enforce good practices and advanced techniques such as concurrency, sharding,
recovery management and transactions.

Some advantages of using a DBMS include:

\begin{itemize}
  \item It decouples data inside a database from the application using it.
        Either can be re-written at any time so long as they still provide/use
        the same interface.
  \item Since the data is decoupled from the application, using a DBMS (in
        theory) lowers the development cost of the application.
  \item Most DBMS are scalable, concurrent, fault tolerant, authorisation
        control (often role based for organisations).
\end{itemize}

Even though the DBMS aims to provide a layer of abstraction for a user
application, there are several layers of abstraction within the DBMS itself.
These are:

\begin{tabularx}{\textwidth}{>{\bfseries}l X}
  Physical & Deals with the file(s) that is written to the
             storage medium that will hold the database. Needs to know about
             file formats, indexing, compression, etc.\\
  Logical  & Mainly concerned with mapping the raw data into database
             `concepts' such as tables, views etc. It is here that the formal
             specification of the database is defined, commonly used models
             include \textit{relational, XML based and document based}\\
  View     & Ensures that only authorised people can view the data.\\
\end{tabularx}

If the database is using a relational format, then it will be defined by a
schema. A schema dictates how the database is formatted; what tables there are,
and what datatypes their columns take. An instance of a database is the content
(data) inside of the database at a particular point in time. There is a certain
isomorphism between relational databases and imperative programming languages; a
schema would be akin to the declaration of variables (i.e. their names and
types), while the instance would be their values at a particular point in the
program's execution.

Irrespective of what logical model a database uses, most DBMS use between one
and three languages to interface with a user/application. These are:

\begin{itemize}
  \item Data Definition Language - used for specifying schema.
  \item Data Manipulation Language - used for mutating the data in the database.
  \item Data Query Language - used to access data in the database.
\end{itemize}

Often DBMS languages will be both a DML and a DQL, and sometimes a DDL too! One
such example is SQL, does all of the above!

\section{Relational Algebra and SQL}

Relational algebra is designed for modelling data stored in relational databases
(i.e. tables) and defining queries on it. It can perform unary operations (such
as growing, shrinking and selecting from tables), or binary operations (union,
intersection, difference, product, join).

\subsection{Selection $\sigma$}

The $\sigma$ operator can select rows that meet a certain criteria from the
table.

\begin{center}
  \begin{tabular}{lll}
    \multicolumn{3}{c}{\textbf{Alcohol-Selection}}\\
    {Type} & {Strength} & {Colour}\\ \hline
    Wine          & 11                & Red\\
    Beer          & 4.2               & Yellow\\
    Wine          & 12.8              & White\\
    Port          & 18                & Carmine\\
    Ale           & 11                & Red\\
  \end{tabular}
\end{center}

If we run $\text{\it Fine-Wines}:=\sigma_{Type=Wine}(\text{\it Alcohol-
Selection})$, we'll end up with:

\begin{center}
  \begin{tabular}{lll}
    \multicolumn{3}{c}{\textbf{Fine-Wines}}\\
    {Type} & {Strength} & {Colour}\\ \hline
    Wine          & 11                & Red\\
    Wine          & 12.8              & White\\
  \end{tabular}
\end{center}

\subsection{Projection $\pi$}

The $\pi$ operator can select rows instead of columns. If we do $\text{\it
Anonymous-Drinks} := \pi_{Strength, Colour}(\text{\it Alcohol-Selection})$:

\begin{center}
  \begin{tabular}{ll}
    \multicolumn{2}{c}{\textbf{Anonymous-Drinks}}\\
    {Strength} & {Colour}\\ \hline
    11         & Red\\
    4.2        & Yellow\\
    12.8       & White\\
    18         & Carmine\\
  \end{tabular}
\end{center}

Notice that both the 11\% Ale and the 11\% Red Wine have the same strength and
colour values. Consequently, the projection operator combines those rows into
one so the same result isn't displayed twice.

Projection can also be used to do simple arithmetic, $\text{\it Test} :=
\pi_{Strength + Strength->DStrength, Colour}(\text{\it Anonymous-Drinks})$:

\begin{center}
  \begin{tabular}{ll}
    \multicolumn{2}{c}{\textbf{Test}}\\
    {DStrength} & {Colour}\\ \hline
    22         & Red\\
    8.4        & Yellow\\
    25.6       & White\\
    36         & Carmine\\
  \end{tabular}
\end{center}

\subsection{Product $\times$}

\begin{center}
  \begin{tabular}{lll}
    \multicolumn{3}{c}{\textbf{Shops}}\\
    {Name}           & {Dodginess} & {Price}\\ \hline
    Ali's            & High        & Medium \\
    Tesco            & Low         & Medium \\
    New Zeland Wines & X.High      & Low    \\
  \end{tabular}
\end{center}

We could do a cross product with the Shops and the Alcohol-Selection tables
$\text{\it Grog-Shops} := \text{\it Shops} \times \text{\it Alcohol-Selection}$:

\begin{center}
  \begin{tabular}{llllll}
    \multicolumn{6}{c}{\textbf{Grog-Shops}}\\
    {Name}           & {Dodginess} & {Price} & {Type} & {Strength} & {Colour}\\ \hline
    Ali's            & High        & Medium  & Wine          & 11                & Red\\
    Ali's            & High        & Medium  & Beer          & 4.2               & Yellow\\
    Ali's            & High        & Medium  & Wine          & 12.8              & White\\
    Ali's            & High        & Medium  & Port          & 18                & Carmine\\
    Ali's            & High        & Medium  & Ale           & 11                & Red\\
    Tesco            & Low         & Medium  & Wine          & 11                & Red\\
    Tesco            & Low         & Medium  & Beer          & 4.2               & Yellow\\
    Tesco            & Low         & Medium  & Wine          & 12.8              & White\\
    Tesco            & Low         & Medium  & Port          & 18                & Carmine\\
    Tesco            & Low         & Medium  & Ale           & 11                & Red\\ 
    New Zeland Wines & X.High      & Low     & Wine          & 11                & Red\\
    New Zeland Wines & X.High      & Low     & Beer          & 4.2               & Yellow\\
    New Zeland Wines & X.High      & Low     & Wine          & 12.8              & White\\
    New Zeland Wines & X.High      & Low     & Port          & 18                & Carmine\\
    New Zeland Wines & X.High      & Low     & Ale           & 11                & Red\\
  \end{tabular}
\end{center}

\subsection{Renaming $\rho$}

The notation for renaming columns is pretty simple; $\text{\it Drinks} :=
\rho_{Name, Strength, Hue}(\text{\it Alcohol-Selection})$

\begin{center}
  \begin{tabular}{lll}
    \multicolumn{3}{c}{\textbf{Drinks}}\\
    {Name} & {Strength} & {Hue}\\ \hline
    Wine          & 11                & Red\\
    Beer          & 4.2               & Yellow\\
    Wine          & 12.8              & White\\
    Port          & 18                & Carmine\\
    Ale           & 11                & Red\\
  \end{tabular}
\end{center}

\subsection{Join $\Join$}

If we had:

\begin{center}
  \begin{tabular}{lll}
    \multicolumn{2}{c}{\textbf{People}}\\
    {Name} & {Drinks}\\ \hline
    Alice  & Wine\\ 
    Bob    & Beer\\ 
  \end{tabular}
\end{center}

We could join it with the Grog-Shops table, using $\text{\it Fave-Shops} :=
People \Join_{People.Drinks=Grog-Shops.Type}(\text{\it Grog-Shops})$

\begin{center}
  \begin{tabular}{lllllll}
    \multicolumn{7}{c}{\textbf{Fave-Shops}}\\
    {Person.Name} & {Grog-Shops.Name}& {Dodginess} & {Price} & {Type} & {Strength} & {Colour}\\ \hline
    Alice         &  Ali's            & High        & Medium  & Wine          & 11                & Red\\
    Bob           &  Ali's            & High        & Medium  & Beer          & 4.2               & Yellow\\
    Alice         &  Ali's            & High        & Medium  & Wine          & 12.8              & White\\
    Alice         &  Tesco            & Low         & Medium  & Wine          & 11                & Red\\
    Bob           &  Tesco            & Low         & Medium  & Beer          & 4.2               & Yellow\\
    Alice         &  Tesco            & Low         & Medium  & Wine          & 12.8              & White\\
    Alice         &  New Zeland Wines & X.High      & Low     & Wine          & 11                & Red\\
    Bob           &  New Zeland Wines & X.High      & Low     & Beer          & 4.2               & Yellow\\
    Alice         &  New Zeland Wines & X.High      & Low     & Wine          & 12.8              & White\\
  \end{tabular}
\end{center}

If two tables have a column of the same name, then they can be joined naturally
without specifying which columns to join explicitly.

\subsection{Distinct $\delta$}

The $\delta$ operator will ensure that no rows are duplicated.

\subsection{Chaining operators}

Just like in normal algebra, you can chain operators:

$\delta(\pi_{Strength, Colour}(\text{\it Alcohol-Selection}))$

Gives:

\begin{center}
  \begin{tabular}{lll}
    {Strength} & {Colour}\\ \hline
    11                & Red\\
    4.2               & Yellow\\
    12.8              & White\\
    18                & Carmine\\
  \end{tabular}
\end{center}

\section{SQL Syntax}

\subsection{Selecting rows}

The SQL command to select rows from a table is:

\begin{verbatim}
	SELECT <column-name>
	FROM <table-name>;
\end{verbatim}

If you only want distinct values from a column, use \texttt{DISTINCT}:

\begin{verbatim}
	SELECT DISTINCT <column-name>
	FROM <table-name>;
\end{verbatim}

If you want all of the columns, use \texttt{*}:

\begin{verbatim}
	SELECT *
	FROM <table-name>;
\end{verbatim}

You can also do derivations:

\begin{verbatim}
	SELECT salary/2
	FROM <table-name>;
\end{verbatim}

In order to specify which rows you want from the table, or you want to join two
tables together, use the \texttt{WHERE} clause:

\begin{verbatim}
	SELECT *
	WHERE <condition>
	FROM <table-name>;
\end{verbatim}

\begin{verbatim}
	SELECT *
	WHERE table1.columnx = table2.columny
	FROM table1, table2;
\end{verbatim}

Use \texttt{AND} to chain multiple conditions in a \texttt{WHERE} clause:

\begin{verbatim}
	SELECT *
	WHERE x > 100
	AND table1.x = table2.y
	FROM <table-name>;
\end{verbatim}

\subsubsection{Where conditions}

You can use the standard $=, <, >$ signs for comparisons between columns,
but SQL also lets you use \texttt{LIKE}, \texttt{BETWEEN} and \texttt{IS NULL}:

\begin{verbatim}
	SELECT *
	FROM <table-name>
	WHERE name LIKE '%BOB%';
\end{verbatim}

\begin{verbatim}
	SELECT *
	FROM <table-name>
	WHERE salary BETWEEN 10000 AND 12000;
\end{verbatim}

\begin{verbatim}
	SELECT *
	FROM <table-name>
	WHERE previous-convictions IS NULL;
\end{verbatim}

You can also compare tuples:

\begin{verbatim}
	SELECT *
	FROM t1, t2
	WHERE (t1.parent, t1.age) = ('Janet', t2.age);
\end{verbatim}

\subsection{Set operations}

The three main set operations are \texttt{UNION, EXCEPT} and \texttt{INTERSECT},
their meaning is rather self evident given their names. An example usage may be:

\begin{verbatim}
	(SELECT *
	 FROM <table-name>)
	UNION ALL
	(SELECT *
	 FROM <table-name>);
\end{verbatim}

\subsection{Joins}

We've already looked at the most common join:

\begin{verbatim}
	SELECT *
	WHERE table1.columnx = table2.columny
	FROM table1, table2;
\end{verbatim}

But you can also do it manually:

\begin{verbatim}
	SELECT *
	FROM table1 JOIN table2
	USING (<column-name>);
\end{verbatim}

Or with a \texttt{NATURAL JOIN}, which is automatic, but can only be applied
when columns have identical names:

\begin{verbatim}
	SELECT *
	FROM table1 NATURAL JOIN table2
\end{verbatim}

\subsection{Renaming columns}

You can use the \texttt{FROM} clause to rename tables:

\begin{verbatim}
	SELECT *
	FROM table1 as a, table2 as b
	where a.col > b.col;
\end{verbatim}

\subsection{Ordering and other operations}

\texttt{GROUP BY} can be used to sort rows by a column value:

\begin{verbatim}
	SELECT *
	FROM <table-name>
	GROUP BY height;
\end{verbatim}

Other operators such as \texttt{AVG} and \texttt{COUNT} can also be used:

\begin{verbatim}
	SELECT AVG(salary)
	FROM <table-name>;	
\end{verbatim}

\begin{verbatim}
	SELECT COUNT(DISTINCT salary)
	FROM <table-name>;	
\end{verbatim}

\begin{verbatim}
	SELECT COUNT(*)
	FROM <table-name>;	
\end{verbatim}

\begin{verbatim}
	SELECT dept_name, AVG(salary)
	FROM staff
	GROUP BY dept_name
	HAVING AVG(salary) > 30000;	
\end{verbatim}

\section{Designing databases}

Entity-Relationship (ER) Modelling is a simple, high-level conceptual modelling
approach that focuses on data requirements rather than business logic. A
conceptual model is what we aim to produce in the design phase of database
development. It describes the format of the database, they types of entity that
are used, the relationships between entities and constraints on the data.

We can translate a conceptual model into a logical schema, which is often just a
language such as SQL. We can use the logical schema directly to create and
manipulate our database.

\subsection{Entity Relationship Modelling}

There are three basic constructs in ER modelling; entity types, attribute types and relationship types.

Attribute types consist of the following:

\begin{description}
  \item Simple or Composite\\
    A simple type is atomic, whereas a composite type is made of an amalgamation
    of multiple other types.
  \item Single or Multi valued\\
    Types can either have single values (such as an integer), or they can have a
    value from discrete set of allowed values (multi-valued).
  \item Stored or Derived\\
    Stored attributes are ones where the value of the attribute is directly
    stored by the database (e.g. licence start date), while Derived attributes
    are computed from the values of other attributes (e.g. licence end date =
    start date + 1 year).
  \item Null valued\\
    Can be null.
  \item Complex-valued\\
    Made of arbitrarily nested composite or multivalued attributes.
\end{description}

\marginpar{Remember, ER models don't have primary or foreign keys, they only
have plain keys. Relationships are modelled explicitly in ER rather than
implicitly using keys.}

Keys are important in ER modelling. Entities with a key are bound by uniqueness
constraints; each key must be unique within the entity. There can be more than
one key in an entity type, but there can also be no key, and if this is the
case, then the entity is said to be \textbf{weak}.

\textbf{Carnality ratio} constraints specify the ratio of one entity to
another in a relationship. It can be $1:1, 1:n, n:1, n:m$.

\textbf{Participation constraints} specify if an entity depends on another
entity. If a participation is \textbf{total}, then every entity in the total
entity must participate in some relationship, if the participation is
\textbf{partial}, then some entities may not participate in any relationship.
For example, student:program is total:partial.

\textbf{Weak entities} can only be properly identified if they are related to
some other entity type, an \textbf{owner}. An \textbf{identifying relationship}
is one where a weak entity has a total participation constraint with it's owner.

When we're writing a requirements specification, \textbf{bold} words are entity
types (e.g. \textbf{employees, departments and projects}), \textit{italic} words
are attribute types (e.g. \textit{name, location, phone number}), and
\underline{underlined} words are relationship types (e.g. employees
\underline{belong to a} department).
