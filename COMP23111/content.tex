% Set the author and title of the compiled pdf
\hypersetup{
  pdftitle = {\Title},
  pdfauthor = {\Author}
}

\section{An introduction to Database Management Systems}

DataBase Management Systems (DBMS's) are a type of middleware that provide a
layer of abstraction for dealing with databases. It is nearly always unnecessary
to write software from scratch that interfaces with a database, since a lot of
database operations will share a significant amount of logic.

Henceforth, a lot of the functionality required of applications that make use of
a database is placed into a DBMS, which application developers can make use and
save time. The DBMS acts as a service, that is well implemented and is able to
enforce good practices and advanced techniques such as concurrency, sharding,
recovery management and transactions.

Some advantages of using a DBMS include:

\begin{itemize}
  \item It decouples data inside a database from the application using it.
        Either can be re-written at any time so long as they still provide/use
        the same interface.
  \item Since the data is decoupled from the application, using a DBMS (in
        theory) lowers the development cost of the application.
  \item Most DBMS are scalable, concurrent, fault tolerant, authorisation
        control (often role based for organisations).
\end{itemize}

Even though the DBMS aims to provide a layer of abstraction for a user
application, there are several layers of abstraction within the DBMS itself.
These are:

\begin{tabularx}{\textwidth}{>{\bfseries}l X}
  Physical & Deals with the file(s) that is written to the
             storage medium that will hold the database. Needs to know about
             file formats, indexing, compression, etc.\\
  Logical  & Mainly concerned with mapping the raw data into database
             `concepts' such as tables, views etc. It is here that the formal
             specification of the database is defined, commonly used models
             include \textit{relational, XML based and document based}\\
  View     & Ensures that only authorised people can view the data.\\
\end{tabularx}

If the database is using a relational format, then it will be defined by a
schema. A schema dictates how the database is formatted; what tables there are,
and what datatypes their columns take. An instance of a database is the content
(data) inside of the database at a particular point in time. There is a certain
isomorphism between relational databases and imperative programming languages; a
schema would be akin to the declaration of variables (i.e. their names and
types), while the instance would be their values at a particular point in the
program's execution.

Irrespective of what logical model a database uses, most DBMS use between one
and three languages to interface with a user/application. These are:

\begin{itemize}
  \item Data Definition Language - used for specifying schema.
  \item Data Manipulation Language - used for mutating the data in the database.
  \item Data Query Language - used to access data in the database.
\end{itemize}

Often DBMS languages will be both a DML and a DQL, and sometimes a DDL too! One
such example is SQL, does all of the above!

\section{Relational Algebra and SQL}

Relational algebra is designed for modelling data stored in relational databases
(i.e. tables) and defining queries on it. It can perform unary operations (such
as growing, shrinking and selecting from tables), or binary operations (union,
intersection, difference, product, join).

\subsection{Selection $\sigma$}

The $\sigma$ operator can select rows that meet a certain criteria from the
table.

\begin{center}
  \begin{tabular}{lll}
    \multicolumn{3}{c}{\textbf{Alcohol-Selection}}\\
    {Type} & {Strength} & {Colour}\\ \hline
    Wine          & 11                & Red\\
    Beer          & 4.2               & Yellow\\
    Wine          & 12.8              & White\\
    Port          & 18                & Carmine\\
    Ale           & 11                & Red\\
  \end{tabular}
\end{center}

If we run $\text{\it Fine-Wines}:=\sigma_{Type=Wine}(\text{\it Alcohol-
Selection})$, we'll end up with:

\begin{center}
  \begin{tabular}{lll}
    \multicolumn{3}{c}{\textbf{Fine-Wines}}\\
    {Type} & {Strength} & {Colour}\\ \hline
    Wine          & 11                & Red\\
    Wine          & 12.8              & White\\
  \end{tabular}
\end{center}

\subsection{Projection $\pi$}

The $\pi$ operator can select rows instead of columns. If we do $\text{\it
Anonymous-Drinks} := \pi_{Strength, Colour}(\text{\it Alcohol-Selection})$:

\begin{center}
  \begin{tabular}{ll}
    \multicolumn{2}{c}{\textbf{Anonymous-Drinks}}\\
    {Strength} & {Colour}\\ \hline
    11         & Red\\
    4.2        & Yellow\\
    12.8       & White\\
    18         & Carmine\\
  \end{tabular}
\end{center}

Notice that both the 11\% Ale and the 11\% Red Wine have the same strength and
colour values. Consequently, the projection operator combines those rows into
one so the same result isn't displayed twice.

Projection can also be used to do simple arithmetic, $\text{\it Test} :=
\pi_{Strength + Strength->DStrength, Colour}(\text{\it Anonymous-Drinks})$:

\begin{center}
  \begin{tabular}{ll}
    \multicolumn{2}{c}{\textbf{Test}}\\
    {DStrength} & {Colour}\\ \hline
    22         & Red\\
    8.4        & Yellow\\
    25.6       & White\\
    36         & Carmine\\
  \end{tabular}
\end{center}

\subsection{Product $\times$}

\begin{center}
  \begin{tabular}{lll}
    \multicolumn{3}{c}{\textbf{Shops}}\\
    {Name}           & {Dodginess} & {Price}\\ \hline
    Ali's            & High        & Medium \\
    Tesco            & Low         & Medium \\
    New Zeland Wines & X.High      & Low    \\
  \end{tabular}
\end{center}

We could do a cross product with the Shops and the Alcohol-Selection tables
$\text{\it Grog-Shops} := \text{\it Shops} \times \text{\it Alcohol-Selection}$:

\begin{center}
  \begin{tabular}{llllll}
    \multicolumn{6}{c}{\textbf{Grog-Shops}}\\
    {Name}           & {Dodginess} & {Price} & {Type} & {Strength} & {Colour}\\ \hline
    Ali's            & High        & Medium  & Wine          & 11                & Red\\
    Ali's            & High        & Medium  & Beer          & 4.2               & Yellow\\
    Ali's            & High        & Medium  & Wine          & 12.8              & White\\
    Ali's            & High        & Medium  & Port          & 18                & Carmine\\
    Ali's            & High        & Medium  & Ale           & 11                & Red\\
    Tesco            & Low         & Medium  & Wine          & 11                & Red\\
    Tesco            & Low         & Medium  & Beer          & 4.2               & Yellow\\
    Tesco            & Low         & Medium  & Wine          & 12.8              & White\\
    Tesco            & Low         & Medium  & Port          & 18                & Carmine\\
    Tesco            & Low         & Medium  & Ale           & 11                & Red\\ 
    New Zeland Wines & X.High      & Low     & Wine          & 11                & Red\\
    New Zeland Wines & X.High      & Low     & Beer          & 4.2               & Yellow\\
    New Zeland Wines & X.High      & Low     & Wine          & 12.8              & White\\
    New Zeland Wines & X.High      & Low     & Port          & 18                & Carmine\\
    New Zeland Wines & X.High      & Low     & Ale           & 11                & Red\\
  \end{tabular}
\end{center}

\subsection{Renaming $\rho$}

The notation for renaming columns is pretty simple; $\text{\it Drinks} :=
\rho_{Name, Strength, Hue}(\text{\it Alcohol-Selection})$

\begin{center}
  \begin{tabular}{lll}
    \multicolumn{3}{c}{\textbf{Drinks}}\\
    {Name} & {Strength} & {Hue}\\ \hline
    Wine          & 11                & Red\\
    Beer          & 4.2               & Yellow\\
    Wine          & 12.8              & White\\
    Port          & 18                & Carmine\\
    Ale           & 11                & Red\\
  \end{tabular}
\end{center}

\subsection{Join $\Join$}

If we had:

\begin{center}
  \begin{tabular}{lll}
    \multicolumn{2}{c}{\textbf{People}}\\
    {Name} & {Drinks}\\ \hline
    Alice  & Wine\\ 
    Bob    & Beer\\ 
  \end{tabular}
\end{center}

We could join it with the Grog-Shops table, using $\text{\it Fave-Shops} :=
People \Join_{People.Drinks=Grog-Shops.Type}(\text{\it Grog-Shops})$

\begin{center}
  \begin{tabular}{lllllll}
    \multicolumn{7}{c}{\textbf{Fave-Shops}}\\
    {Person.Name} & {Grog-Shops.Name}& {Dodginess} & {Price} & {Type} & {Strength} & {Colour}\\ \hline
    Alice         &  Ali's            & High        & Medium  & Wine          & 11                & Red\\
    Bob           &  Ali's            & High        & Medium  & Beer          & 4.2               & Yellow\\
    Alice         &  Ali's            & High        & Medium  & Wine          & 12.8              & White\\
    Alice         &  Tesco            & Low         & Medium  & Wine          & 11                & Red\\
    Bob           &  Tesco            & Low         & Medium  & Beer          & 4.2               & Yellow\\
    Alice         &  Tesco            & Low         & Medium  & Wine          & 12.8              & White\\
    Alice         &  New Zeland Wines & X.High      & Low     & Wine          & 11                & Red\\
    Bob           &  New Zeland Wines & X.High      & Low     & Beer          & 4.2               & Yellow\\
    Alice         &  New Zeland Wines & X.High      & Low     & Wine          & 12.8              & White\\
  \end{tabular}
\end{center}

If two tables have a column of the same name, then they can be joined naturally
without specifying which columns to join explicitly.

\subsection{Distinct $\delta$}

The $\delta$ operator will ensure that no rows are duplicated.

\subsection{Chaining operators}

Just like in normal algebra, you can chain operators:

$\delta(\pi_{Strength, Colour}(\text{\it Alcohol-Selection}))$

Gives:

\begin{center}
  \begin{tabular}{lll}
    {Strength} & {Colour}\\ \hline
    11                & Red\\
    4.2               & Yellow\\
    12.8              & White\\
    18                & Carmine\\
  \end{tabular}
\end{center}

\section{SQL Syntax}

\begin{mymulticols}

  \subsection{Selecting rows}

  The SQL command to select rows from a table is:

  \begin{alltt}
  	SELECT <column-name>\\
  	FROM <table-name>;
  \end{alltt}

  If you only want distinct values from a column, use \texttt{DISTINCT}:

  \begin{alltt}
  	SELECT DISTINCT <column-name>\\
  	FROM <table-name>;
  \end{alltt}

  If you want all of the columns, use \texttt{*}:

  \begin{alltt}
  	SELECT *\\
  	FROM <table-name>;
  \end{alltt}

  You can also do derivations:

  \begin{alltt}
  	SELECT salary/2\\
  	FROM <table-name>;
  \end{alltt}

  In order to specify which rows you want from the table, or you want to join two
  tables together, use the \texttt{WHERE} clause:

  \begin{alltt}
  	SELECT *\\
  	WHERE <condition>\\
  	FROM <table-name>;
  \end{alltt}

  \begin{alltt}
  	SELECT *\\
  	WHERE table1.columnx = table2.columny\\
  	FROM table1, table2;
  \end{alltt}

  Use \texttt{AND} to chain multiple conditions in a \texttt{WHERE} clause:

  \begin{alltt}
  	SELECT *\\
  	WHERE x > 100\\
  	AND table1.x = table2.y\\
  	FROM <table-name>;
  \end{alltt}

  \subsubsection{Where conditions}

  You can use the standard $=, <, >$ signs for comparisons between columns,
  but SQL also lets you use \texttt{LIKE}, \texttt{BETWEEN} and \texttt{IS NULL}:

  \begin{alltt}
  	SELECT *\\
  	FROM <table-name>\\
  	WHERE name LIKE '%BOB%';
  \end{alltt}

  \begin{alltt}
  	SELECT *\\
  	FROM <table-name>\\
  	WHERE salary BETWEEN 10000 AND 12000;
  \end{alltt}

  \begin{alltt}
  	SELECT *\\
  	FROM <table-name>\\
  	WHERE previous-convictions IS NULL;
  \end{alltt}

  You can also compare tuples:

  \begin{alltt}
  	SELECT *\\
  	FROM t1, t2\\
  	WHERE (t1.parent, t1.age) = ('Janet', t2.age);
  \end{alltt}

  \subsection{Set operations}

  The three main set operations are \texttt{UNION, EXCEPT} and \texttt{INTERSECT},
  their meaning is rather self evident given their names. An example usage may be:

  \begin{alltt}
  	(SELECT *\\
  	 FROM <table-name>)\\
  	UNION ALL\\
  	(SELECT *\\
  	 FROM <table-name>);
  \end{alltt}

  \subsection{Joins}

  We've already looked at the most common join:

  \begin{alltt}
  	SELECT *\\
  	WHERE table1.columnx = table2.columny\\
  	FROM table1, table2;
  \end{alltt}

  But you can also do it manually:

  \begin{alltt}
  	SELECT *\\
  	FROM table1 JOIN table2\\
  	USING (<column-name>);
  \end{alltt}

  Or with a \texttt{NATURAL JOIN}, which is automatic, but can only be applied
  when columns have identical names:

  \begin{alltt}
  	SELECT *\\
  	FROM table1 NATURAL JOIN table2
  \end{alltt}

  \subsection{Renaming columns}

  You can use the \texttt{FROM} clause to rename tables:

  \begin{alltt}
  	SELECT *\\
  	FROM table1 as a, table2 as b\\
  	where a.col > b.col;
  \end{alltt}

  \subsection{Ordering and other operations}

  \texttt{GROUP BY} can be used to sort rows by a column value:

  \begin{alltt}
  	SELECT *\\
  	FROM <table-name>\\
  	GROUP BY height;
  \end{alltt}

  Other operators such as \texttt{AVG} and \texttt{COUNT} can also be used:

  \begin{alltt}
  	SELECT AVG(salary)\\
  	FROM <table-name>;	
  \end{alltt}

  \begin{alltt}
  	SELECT COUNT(DISTINCT salary)\\
  	FROM <table-name>;	
  \end{alltt}

  \begin{alltt}
    SELECT COUNT(*)\\
    FROM <table-name>;
  \end{alltt}

  \begin{alltt}
    SELECT dept-name, AVG(salary)\\
    FROM staff\\
    GROUP BY dept-name\\
    HAVING AVG(salary) > 30000;
  \end{alltt}

\end{mymulticols}

\section{Designing databases}

Entity-Relationship (ER) Modelling is a simple, high-level conceptual modelling
approach that focuses on data requirements rather than business logic. A
conceptual model is what we aim to produce in the design phase of database
development. It describes the format of the database, they types of entity that
are used, the relationships between entities and constraints on the data.

We can translate a conceptual model into a logical schema, which is often just a
language such as SQL. We can use the logical schema directly to create and
manipulate our database.

\subsection{Entity Relationship Modelling}

There are three basic constructs in ER modelling; entity types, attribute types
and relationship types.

Attribute types consist of the following:

\begin{description}
  \item Simple or Composite\\
    A simple type is atomic, whereas a composite type is made of an amalgamation
    of multiple other types.
  \item Single or Multi valued\\
    Types can either have single values (such as an integer), or they can have a
    value from discrete set of allowed values (multi-valued).
  \item Stored or Derived\\
    Stored attributes are ones where the value of the attribute is directly
    stored by the database (e.g. licence start date), while Derived attributes
    are computed from the values of other attributes (e.g. licence end date =
    start date + 1 year).
  \item Null valued\\
    Can be null.
  \item Complex-valued\\
    Made of arbitrarily nested composite or multivalued attributes.
\end{description}

\marginpar{Remember, ER models don't have primary or foreign keys, they only
have plain keys. Relationships are modelled explicitly in ER rather than
implicitly using keys.}

Keys are important in ER modelling. Entities with a key are bound by uniqueness
constraints; each key must be unique within the entity. There can be more than
one key in an entity type, but there can also be no key, and if this is the
case, then the entity is said to be \textbf{weak}.

\textbf{Carnality ratio} constraints specify the ratio of one entity to
another in a relationship. It can be $1:1, 1:n, n:1, n:m$.

\textbf{Participation constraints} specify if an entity depends on another
entity. If a participation is \textbf{total}, then every entity in the total
entity must participate in some relationship, if the participation is
\textbf{partial}, then some entities may not participate in any relationship.
For example, student:program is total:partial.

\textbf{Weak entities} can only be properly identified if they are related to
some other entity type, an \textbf{owner}. An \textbf{identifying relationship}
is one where a weak entity has a total participation constraint with it's owner.

When we're writing a requirements specification, \textbf{bold} words are entity
types (e.g. \textbf{employees, departments and projects}), \textit{italic} words
are attribute types (e.g. \textit{name, location, phone number}), and
\underline{underlined} words are relationship types (e.g. employees
\underline{belong to a} department).

\subsubsection{ER Notation}

\begin{mymulticols}

  \begin{description}
    \item \textbf{Entity type}\\
    \begin{center}
      \begin{tikzpicture}[node distance=2.5cm, every edge/.style={link}]
        \node[entity] {Employee};
      \end{tikzpicture}
    \end{center}

    \item \textbf{Weak entity type}\\
    \begin{center}
      \begin{tikzpicture}[node distance=2.5cm, every edge/.style={link}]
        \node[weak entity] {Dependent};
      \end{tikzpicture}
    \end{center}

    \item \textbf{Attribute}\\
    \begin{center}
      \begin{tikzpicture}[node distance=2.5cm, every edge/.style={link}]
        \node[entity] (emp) {Employee};
        \node[attribute] [above=1cm of emp] {Name} edge (emp);
      \end{tikzpicture}
    \end{center}

    \item \textbf{Weak attribute}\\
    \begin{center}
      \begin{tikzpicture}[node distance=2.5cm, every edge/.style={link}]
        \node[weak entity] (emp) {Dependent};
        \node[attribute] [above=1cm of emp] {\discriminator{Name}} edge (emp);
      \end{tikzpicture}
    \end{center}

    \item \textbf{Derived attribute}\\
    \begin{center}
      \begin{tikzpicture}[node distance=1.5cm, every edge/.style={link}]
        \node[entity] (emp) {Employee};
        \node[attribute] [above=1cm of emp] {Date of Birth} edge (emp);
        \node[derived attribute] [left= 1cm of emp] {Age} edge (emp);
      \end{tikzpicture}
    \end{center}

    \item \textbf{Multivalue}\\
    \begin{center}
      \begin{tikzpicture}[node distance=2.5cm, every edge/.style={link}]
        \node[entity] (emp) {Employee};
        \node[multi attribute] [above=1cm of emp] {Children} edge (emp);
      \end{tikzpicture}
    \end{center}

    \item \textbf{Key}\\
    \begin{center}
      \begin{tikzpicture}[node distance=2.5cm, every edge/.style={link}]
        \node[entity] (emp) {Employee};
        \node[attribute] [above=1cm of emp] {\key{ID}} edge (emp);
      \end{tikzpicture}
    \end{center}

    \item \textbf{Relationship (plus carnality ratio)}\\
    \begin{center}
      \begin{tikzpicture}[node distance=2.5cm, every edge/.style={link}]
        \node[entity] (emp) {Employee};
        \node[relationship] [above=1cm of emp] (rel) {Works for};
        \node[entity] [above=1cm of rel] (dep) {Department};
        \draw[link] (rel.90) |- node [below, auto] {1} (dep.270);
        \draw[link] (rel.270) |- node [above, auto] {N} (emp.90);
      \end{tikzpicture}
    \end{center}

    \item \textbf{Identifying relationship}\\
    \begin{center}
      \begin{tikzpicture}[node distance=2.5cm, every edge/.style={link}]
        \node[entity] (emp) {Employee};
        \node[ident relationship] [above=1cm of emp] (rel) {Dependent of} edge (emp);
        \node[weak entity] [above=1cm of rel] (dep) {Dependent} edge (rel);
      \end{tikzpicture}
    \end{center}

    \item \textbf{Recursive relationship}\\
    \begin{center}
      \begin{tikzpicture}[node distance=2.5cm, every edge/.style={link}]
        \node[entity] (emp) {Employee};
        \node[relationship] [below=1cm of emp] (rel) {Dependent of};
        \draw[link] (rel.180) |- node [left, auto] {Supervisor} (emp.180);
        \draw[link] (emp) -| node [left, auto] {Supervisee} (rel.360);
      \end{tikzpicture}
    \end{center}

    \item \textbf{Participation constraint}\\
    \begin{center}
      \begin{tikzpicture}[node distance=2.5cm, every edge/.style={link}]
        \node[entity] (emp) {Project};
        \node[relationship] [above=1cm of emp] (rel) {Controls} edge [total] (emp);
        \node[entity] [above=1cm of rel] (dep) {Department} edge (rel);
      \end{tikzpicture}
    \end{center}
  \end{description}
\end{mymulticols}

\subsubsection{Specialising entities}

It's often efficient to refine entities into sub-entities if there are fields
that are relevant for some members and not others. For example a vehicle entity
could have car and lorry sub-entities. One way of doing this is top-down
conceptual refinement, which is a fancy way of saying, start with an entity,
identify it's subclasses, create sub-entities for them, and then repeat for the
newly created sub-entities.

\marginpar{I've not got time to work out how to do $\cup$ arrowheads now, please
draw them on the diagram!}

\begin{center}
  \begin{tikzpicture}[node distance=2.5cm, every edge/.style={link}]
    \node[entity] (emp) {Vehicle};
    \node[attribute] [below=1cm of emp] (rel) {d} edge (emp);
    \node[entity] [below left=1cm of rel] (dep) {Lorry} edge (rel);
    \node[entity] [below right=1cm of rel] (dep2) {Car} edge (rel);
  \end{tikzpicture}
\end{center}

An alternate way to specialise is to do a bottom-up approach. Start with lots of
sets, and define a common superset of them. Here, you're generalising rather
than specialising.

Each subentity can have it's own attributes and relationships of course:

\begin{center}
  \begin{tikzpicture}[node distance=2.5cm, every edge/.style={link}]
    \node[entity] (emp) {Vehicle};
    \node[attribute] [above left=1cm of emp] {Colour} edge (emp);
    \node[attribute] [above right=1cm of emp] {Model} edge (emp);
    \node[attribute] [below=1cm of emp] (rel) {d} edge (emp);
    \node[entity] [below left=1cm of rel] (dep) {Lorry} edge (rel);
    \node[entity] [below right=1cm of rel] (dep2) {Car} edge (rel);
    \node[attribute] [below right=1cm of dep2] {\# Doors} edge (dep2);
    \node[attribute] [below left=1cm of dep2] {\# Seats} edge (dep2);
    \node[relationship] [below =1cm of dep] (rel2) {Registered to} edge (dep);
    \node[entity] [below right=1cm of rel2] {Country} edge (rel2);
  \end{tikzpicture}
\end{center}

Entity subtypes can be user, attribute or predicate defined based on whether the
user has defined them to be a subtype, their attributes define them to be
subtypes or if a predicate has (e.g. if country in EU $\implies$ EU Member).

In the examples, the attribute `d' is used to create a subtype. This stands for
disjoint, since a car can't also be a lorry. There is also `o', which is used
when an entity could belong to multiple subtypes.

\section{Converting an ER diagram into a relational language}

I refer you to Alvaro's sixth lecture, for this part, since it'd take too long
to draw all the diagrams you need. If you're reading this and feeling helpful,
you write this section and send me a pull request!

\section{Normalization}

Normalizeation is the process of decomposing unsatisfactory (i.e. potentially
improvable) relations by creating smaller relations from them. The keys and ???
of the relation determine it's normal form.

\marginpar{3NF and BCNF are the targets we try and work towards.}

The normal form indicates a particular quality level of the database schema. 1NF
is a relation in its first normal form, where every field contains only atomic
values. 2NF, 3NF and BCNF are normal forms that are defined in terms of keys and
??? of a relational schema.

The main technique that we use to try and refine schema is
\textbf{decomposition}. This is the idea of taking a relation ABCD, and
decomposing it into two relations; AB and BCD. Decomposition is guided by the
functional dependencies that hold over the relation.

An example would be a table of Employees. If there was a column for office id
and another for work address, we would have to update all of the employees who
worked at a specific office if the office location moved. If there was a
seperate table of offices, we could simply update the office location in that
table, and because of a relation between Employee and Office, we wouldn't have
to update Employee at all!
