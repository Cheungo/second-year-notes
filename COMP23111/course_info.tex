\section*{Introduction}

Databases are core, if largely invisible, components of modern computing
architectures in both commercial and scientific contexts. The management of data
has evolved from application-specific management of myriad files to
organisation-wide approaches that see data as one of the most important assets
of modern organisations and, as such, a key factor in their ability to compete
and thrive. At this organisation-wide scale, database management systems (DBMSs)
are the crucial piece of software infrastructure needed to achieve the desired
results with consistent quality and robust efficiency. A modern DBMS is a thing
of wonder and embodies in its internal construction and in its wide usability
many advances in algorithms and data structures, programming language theory,
conceptual modelling, concurrency theory, and distributed computing. This makes
the study of databases a data-centric traversal of many of the most exciting
topics in modern computing.

\section*{Aims}

The aim of this course unit is to introduce the students to the fundamental
concepts and techniques that underlie modern database management systems
(DBMSs).

The course unit studies the motivation for managing data as an asset and
introduces the basic architectural principles underlying modern DBMSs. Different
architectures are considered and the application environments they give rise to.

The course unit then devotes time to describing and motivating the relational
model of data, the relational database languages, and SQL, including views,
triggers, embedded SQL and procedural approaches (e.g., PL/SQL).

The students learn how to derive a conceptual data model (using the Extended
Entity Relationship paradigm), how to map such a model to target implementation
model (for which the relational model is used), how to assess the quality of the
latter using normalisation, and how to write SQL queries against the improved
implementation model to validate the resulting design against the data
requirements originally posed. For practical work, the Oracle DBMS is used.

The course unit also introduces the fundamentals of transaction management
including concurrency (e.g., locking, 2-phase locking, serialisability) and
recovery (rollback and commit, 2-phase commit) and of file organisation (e.g.,
clustering) and the use of indexes for performance.

Finally, the course unit addresses the topic of database security by a study of
threats and countermeasures available. In the case of the former, these include
potential theft and fraud as well as loss of confidentiality, privacy, integrity
and availability. In the case of the latter these primarily include mechanisms
for authorization and access control, including the use of views for that
purpose. The course unit also addresses the topic of legal frameworks that give
rise to obligations on the part of database professionals by introducing
exemplars such as the 1995 EU Directive on Data Protection and the 1998 UK Data
Protection Act.

\section*{Additional reading}

%TODO: Add reading