\section*{Introduction}

Operating systems provide an interface for computer users that permits them to
gain access without needing to understand how the computer works. The software
needed to achieve this is complex and this course introduces students to some of
the details of design and implementation.

\section*{Aims}

This course unit introduces students to the principles of operating system
design and to the prevailing techniques for their implementation. The course
unit assumes that students are already familiar with the structure of a user-
program after it has been converted into an executable form, and that they have
a rudimentary understanding of the performance trade-offs inherent in the choice
of algorithms and data structures. Pertinent features of the hardware-software
interface are described, and emphasis is placed on the concurrent nature of
operating system activities. Two concrete examples of operating systems are used
to illustrate how principles and techniques are deployed in practice.

\section*{Additional reading}

\begin{tabularx}{\textwidth}{X|X|l}
  Operating system concepts (8th edition) & Silberschatz, Abraham and Peter Baer Galvin and Greg Gagne & 2009\\ \hline
  Operating system concepts with Java (8th edition) & Silberschatz, Abraham and Peter Baer Galvin and Greg Gagne & 2010
\end{tabularx}