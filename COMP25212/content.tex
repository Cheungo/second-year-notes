% Set the author and title of the compiled pdf
\hypersetup{
	pdftitle = {\Title},
	pdfauthor = {\Author}
}

\section{Introduction}

Performance is always an attribute in high demand in computer systems. Even
though processors have become so much more powerful over the last half century,
there's still loads of stuff that we cannot do with current technology, such as
synthesising HD video in realtime, or computing realistic game physics.

Since 2004/5, companies haven't been able to increase the speed of
microprocessors at such a rapid rate due to physical limits, such as power
dissipation and device variability. Our devices are still getting faster, but
now architecture and the design of systems play a larger role in making stuff
run faster. An example of this include making computation more parallel.

\section{Caches}

Not all technology has improved at the same relative speed. CPU's have become
over three orders of magnitude faster over the past thirty years, but memory has
increased by only one order of magnitude. This is problematic, since it means
that we need to reconcile this gap in order to achieve efficient computation.

Processor caching is used to let the processor do useful computation while
it's also waiting on the memory. Modern processors couldn't perform anywhere
near how fast they do now without equally modern caching techniques, since the
imbalance between the CPU and main memory is so high.

Caches (in general) provide a limited, but very fast local space for the CPU to
use. They are used in lots of places all over computer science, including web
browsers, mobile phone UI's etc. Likewise, a processor cache is a temporary
store for frequently used memory locations.

The principle of locality is what makes caches work for processors, which is
that the CPU will only use a small subset of memory over a short period of time.
If this subset of memory can be loaded into the cache, then the computation can
be sped up significantly.

Every `cache miss' takes \textit{at least} sixty times longer to execute than a
`cache hit' will (that's assuming there are no page faults etc). Circuit
capacitance is the thing that makes electronic devices slow, and larger
components have a larger capacitance, henceforth large memories are slow.
Dynamic memories (DRAM) store data using capacitance, and are therefore slower
than static memories (SRAM) that work using bistable circuits.

Even the wires between the processor and the memory have a significant
capacitance. Driving signals between chips needs specialised high power
interface circuits. An ideal situation would be to have everything on a single
chip, however current manufacturing limitations prevent this; maybe one day we
will be able to do this.

% TODO: Memory heirachy diagram

\subsection{Why are caches expensive?}

L1, L2 and (usually) L3 caches are SRAM instead of DRAM (which is what main
memory is made from).

SRAM needs six transistors per bit, DRAM needs one.

SRAM is henceforth physically larger, taking up more space on the chip, which is
expensive, since real estate costs money.


\subsection{L1 Cache}

The L1 cache is the first level of caching between the processor and the main
memory. The L1 cache is around 32kb, which is very small in comparison to the
size of the main memory, but this is driven out of necessity, since the cache
needs to be small to be fast. The cache must be able to hold any arbitrary
location in the main memory (since we don't know in advance what the CPU will
want), and henceforth requires specialised structures to implement this.


\subsection{Types of cache}

The cpu will give the cache a full address of a word in memory that it wants to
read. The cache will contain a small selection of values from memory that it has
locally, but will ask the main memory for values that it does not have. This is
called a cache miss and is expensive in comparison to a cache hit.

\subsubsection{Fully associative}

A \textbf{Fully Associative} cache is one where the cache is small (around
32,000 values), but stores both addresses and their corresponding data. The
hardware compares the input address with all of the stored addresses (it does
this in parallel). If the address is found, then the value is returned with no
need to ask the RAM (cache hit), if the value isn't found, then a cache miss
occurs, and the request must go to the main memory.

Caches rely on locality in order to function effectively. There are two types of
locality; temporal locality, which is the principle that if you use an address
once, you may use it again soon (e.g. loops), and spatial locality, where if you
use an address once, you are also likely to use addresses nearby (e.g. arrays).

\marginpar{Spatial locality is exploited better by having a bigger data area in
the cache (returning say 512 bits for every address instead of just one word)}

The cache hit rate is the ratio of cache hits to misses. We need a hit rate of
$98\%$ to hide the speed of memory from the CPU. Instruction hit rates are
usually better than data hit rates (although they are in the same cache
remember). The reason for this is that instructions are accessed in a more
regular pattern, incrementing by one word every time, or looping around etc
(have higher locality).

When we do a cache miss (read), we should add the missed value to the cache. In
order to do this, we need a cache replacement policy to find room in the cache
to put the new value:

\begin{itemize}
	\item LRU - slow, good for hit rates
	\item Round Robin - not as good, easier to implement
	\item Random - Easy to implement, works better than expected.
\end{itemize}

Memory writes are more complicated than reads. If we've already got the value in
the cache, then we change the value in the cache. We can use three write
strategies for cache writes (on hits):

\begin{itemize}
	\item Write through (slow)
	\item Write through + buffer (faster, slow when heavily used)
	\item Copy back on cache replacement.
\end{itemize}

On misses:

\begin{itemize}
	\item We can find a location in the cache, and write to that, then rely on
	copy back later, or write back straight away.
	\item we can skip the cache, and write directly to RAM. Subsequent read will
	add to the cache if necessary (good if you're initialising datastructures
	with zeroes).
\end{itemize}

Fastest one is write allocate or copy back. Main memory and the cache aren't
coherent, which can be a problem for stuff like multiprocessors, autonomous IO
devices etc. This may need cleaning up later.

Each cache line is at least half address and half data, but often, we store more
data per address, so will have 64 bytes of data per 32 bit address.

A fully associative cache is ideal, but this is expensive (in terms of silicon
and power).

\subsubsection{Directly mapped}

We can use standard RAM to create a directly mapped cache, which mimics the
functionality of an ideal cache. Usually, this uses static RAM, which is more
expensive than dynamic RAM, but is faster. The address is divided into two
parts, %TODO: Directly mapepd cache 2,3,4 + direct mapped replacement?

\subsubsection{Set associative}

Set associative caches are a compromise. They comprise of a number of directly
mapped caches operating in parallel. If one matches, we have a hit and select
the appropriate data. This is good because we can have more flexible cache
replacement strategies. In a 4 way, we could choose any one of the four caches
for example. The hit rate of set associative caches improves with the number of
caches, but increasing the number increase the cost.

% Lecture three

\section{Practical caches}

\subsection{Cache control bits}

When the system is started, the cache is empty. We need a bit for each cache
entry to indicate that the data is meaningful (i.e. it isn't just an unitialised
zero or something). We also need a dirty bit if we're using the `write back'
caching strategy (see above), rather than the `write through' strategy.

\subsection{Exploiting spatial locality}

In order to exploit spatial locality, we need to have a wider `cache line', where
each entry will give you more data than just one word. Each entry tag could
correspond to two, four, eight etc words. Spatial locality says that if we get
one byte, we'll probably want one from close by too.

% TODO: Make image

The lowest bits are used to select the word in the cache line. Most cache lines
are 16 or 32 bytes, which is 4 or 8 32bit words. The data is transferred from
RAM in bursts equal to the width of the line size, using specialised memory
access modes.

%TODO: Explain about more cache line = less number of tags? Is that even right?

The line size is important, since we want to have a line size of multiple words
to exploit spatial locality, but if the line is too big, then parts of it will
never be used. The number of cache misses decreases as you increase the cache
line size, until one point, where the line size will be too long to use all the
words, and then the number of misses will increase.

\subsection{Separate instruction and data caches}

Since instructions and data have different access patterns in memory (but they
are stored in the same memory), we could use different caches for each type of
word, so that the different caches can use different strategies to minimise
misses according to their different access patterns.

\subsection{Multi level caches}

As chips get bigger, in theory, we should build bigger caches to perform better.
However, big caches are slow, and the L1 cache needs to run at processor speed.
We can instead put another cache between the RAM and the L1 cache, and keep the
L1 cache the same size.

The L2 cache is typically sixteen times bigger than the L1 cache, but also four
times slower. It's still ten times faster than RAM though. The L1D and L1I
caches both share the L2 cache.

If a chip has an L3 cache, then it is usually quite large (maybe around 8Mb),
but its performance is only about twice as good as that of RAM.

%TODO: Cache example from slide 16

\subsection{Cache misses}

There are three types of cache misses, called the three C's

\begin{description}
  \item \textbf{Compulsory misses}\\ 
    When we first start the computer, the cache is empty, so until the cache is
    populated, we're going to have a lot of misses.
  \item \textbf{Capacity misses}\\
   Since the cache is limited in size, we can't contain all of the pages for a
   program, so some misses will occur.
  \item \textbf{Conflict misses}\\
    In a direct mapped or set associative cache, there is competition between
    memory locations for places in the cache. If the cache was fully
    associative, then misses due to this wouldn't occur.
\end{description}

\subsection{More cache performance}

%TODO: What happens when IO is memory mapped, does this go through the cache? :o

In order to fill a cache from empty, it takes $\frac{\text{Cache
size}}{\text{Line size}}$ memory accesses. If we multiply this by the time it
takes for a single memory access (say $10\micro\second$), then we can work out
how long it will take to fill the cache (assuming each access is to a unique
memory address). We can derive how many CPU cycles this takes from this.

\subsection{Cache consistency}

We need to make sure that the values stored in the CPU cache are consistent with
those in main memory. There are situations when they can disagree, for example
if IO reads or writes directly to memory (perhaps using DMA), then that value
could be different from whatever is in the cache.

Solutions:
\begin{description}
  \item \textbf{Non-cacheable}\\
    One solution is to make areas of memory that IO can access non-cachable, or
    clear the cache before and after the IO takes place.
  \item \textbf{IO use data cache}\\
    Another is to have the IO go directly through the CPU's L1d (data) cache
    before accessing memory, but this is slow.
  \item \textbf{Snoop on IO activity}\\
    We could have hardware logic that will look at the reads and writes to
    memory from IO and make sure the cache is consistent with memory for those
    addresses.
\end{description}

\subsection{Virtual Addresses}

Since the CPU deals with virtual addresses when accessing memory, and uses a
Translation Lookaside Buffer to derive the correct physical address. However,
which address does the cache store? Does it sit before the TLB, or after it
between the CPU and memory?

If we make addresses go through the TLB before they reach the cache, then this
is slow, since they must pass through extra logic etc before hitting the cache.
However, if we make the cache store virtual addresses, and have the TLB sit
inbetween the cache and memory, this makes snooping hard to implement along with
more functional difficulties.

The answer is to have the TLB operate in parallel to the cache. Since address
translation only affects the high order bits of the cache (the low order bits
are the offset which remains the same). The cache index is selected from the low
order offset bits, and only the tag is changed by address translation.

%TODO: Image goes here (something like in the lecture slide of lecture 4, S15)

\section{Pipelines}

\section{Multi-Threading}

\section{Multi-Core}

\section{Vitalisation}

\section{Permanent Storage}
