\card{
	What does a fully associative cache store?
}{
	Addresses and their corresponding data
}

\card{How does the CPU locate an item in a fully associative cache?}
{Hardware compares the input address with all stored addresses (in parallel)\\
If we get a match we have a hit\\
If no match we must go to main memory}

\card{What is temporal locality?}
{The principle that if you use an address once, you may use it again soon e.g. loops}

\card{What is spatial locality?}
{The principle that if you use an address once, you are also likely to use addresses nearby e.e. arrays}

\card{What are the three common cache replacement algorithms?}
{Least Recently Used (LRU)\\
Round Robin\\
Random}

\card{Explain the write-through cache write strategy.}
{Whenever a write is done to the cache, the write is also done to main memory}

\card{Explain the copy-back cache write strategy.}
{When a cach line is replaced, if the dirty bit is set, the modified value is written to main memory}

\card{Why does a direct mapped cache usually use static RAM?}
{It is a lot faster than dynamic RAM}

\card{Briefly explain what a set associative cache consists of.}
{A number of directly mapped caches operating in parallel}

\card{What is the advantage of using a set associative cache?}
{We have more flexible cache replacement strategies as we could choose any one of the caches to replace from}

\card{What two control bits are usually used in cache entries?}
{Valid bit and dirty bit}

\card{Explain what a compulsory cache miss is?}
{When we first start the computer, the cache is empty, so until the cache is populated, we're going to have a lot of misses}

\card{Explain what a capacity cache miss is?}
{Since the cache is limited in size, we can’t contain all of the pages for a program, so some misses
will occur.}

\card{Explain what a conflict cache miss is?}
{In a direct mapped or set associative cache, there is competition between memory locations for
places in the cache. If the cache was fully associative, then misses due to this wouldn’t occur.}