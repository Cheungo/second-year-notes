% Set the author and title of the compiled pdf
\hypersetup{
	pdftitle = {\Title},
	pdfauthor = {\Author}
}

\section{Algorithmic complexity and performance}

Algorithmic complexity and the big-oh notation allows us to characterise the
time and space requirements of an algorithm when it is given varying input data.
The big-oh notation allows us to get a good approximation of the upper and lower
bounds of an algorithm's complexity.

We can work out such an approximation by analysing (and generalising) the number
of logical operations an algorithm might do, rather than inspecting it's
performance in an implementation. This allows us to compare the merits of
different algorithms irrespective of their implementation.

The big-oh notation is shown below:

\[
  O(growth rate)
\]

The growth rate represents the rate at which the complexity of the algorithm
will change with the size of the input.

Growth rates that are either exponential or factorial in nature (or are perhaps
even worse than this) are said to be intractable \marginpar{Tractable
(\textit{Adjective})\\Easy to deal with.}, while algorithms with other
computational complexities are said to be tractable.

\newcommand\multibrace[3]{\rdelim\}{#1}{3mm}[\pbox{#2}{#3}]}

\begin{table}[h!]
  \centering
  \begin{tabularx}{0.75\textwidth}{>{$}l<{$} l l}
    \text{Complexity} & Growth rate \\ \cline{1-2}
    O(1)              & None\\
    O(log n)          & Logarithmic\\
    O(n^k)            & Polynomial\\
    O(n)              & Linear & \multibrace{3}{4.6cm}{
                                  All of these are special cases of polynomials,
                                  $n^1, n^2$ and $n^3$ respectively
                                } \\
    O(n^2)            & Quadratic\\
    O(n^3)            & Cubic\\
    O(k^n)            & Exponential\\
    O(n!)             & Factorial\\
  \end{tabularx}
  \caption{A number of common complexities and their equivalent growth rates}
  \label{table:complexity}
\end{table}

\subsection{Simplifying Big-Oh expressions}

In order to simplify the big-oh complexity of an algorithm you just isolate the
fastest growing term in the equation (i.e. whatever term comes furthest down in
Table~\ref{table:complexity}). You then remove all the constants from the
equation.

%TODO: Big Oh Example! Ohh no!

\subsection{Analysing algorithmic complexity}

There are two ways of finding the complexity of an algorithm, to inspect the
psudo code for it, or by implementing the algorithm and experimentally
determining the change in it's runtime with different input sizes.

\marginpar{Remember to check that your psudo code correctly implements the
algorithm before you try this.}

In order to analyse the psudo code to work out the complexity, you must look at
how many primitive operations it will use for different sizes of input.
Primitive operations are defined as memory accesses, arithmetic operations,
comparisons and the like.

As a general rule, loops, recursion and other constructs for repeatedly
performing operations will the best indicator as to the complexity of
algorithms.

If an algorithm is reducing the data it has to work with every so often, then it
may have a logarithmic runtime. For example, the algorithm utilises a binary
chop (such as binary search), then the runtime probably has a $log_2$ inside.
See section~\ref{subsubsec:logs} for a bit more on logs.

In order to determine complexity experimentally, you must implement the
algorithm in a programming language of your choice, then run the program for
different input sizes and measure the runtime and the memory used. Plot the data
on a graph and extrapolate as needed. From the curve of the graph, it is
possible to predict the complexity of the algorithm.

\subsubsection{A refresher on logs}
\label{subsubsec:logs}

A logarithm of a number is the exponent to which another number (the base) must
be raised to produce that number:

\begin{gather*}
  \forall b,x,y \in \mathbb{Z}\\
  y = b^x \Leftrightarrow x = log_b(y)
\end{gather*}

Henceforth, $2^4 = 16$ and $log_2(16) = 4$.

\subsubsection{Finding the maximum input size}

If we know the complexity and running space/time of an algorithm for a specific
implementation and input, we might want to know the input size we could run the
algorithm with for the specific machine in a specific time, or within a specific
space limit.

The way you do this is by solving the big-oh equation for $t$ instead of $n$.
For example, if the algorithm takes $30$ seconds to process $1000$ kilobytes of
data, how long will it take if we double the processing speed, given that the
algorithm runs in $O(n^3)$ time?

%TODO: Work out if this is BS or not?

\begin{gather*}
  t = n^3\\
  \sqrt[\leftroot{-0}\uproot{3}3]{t} = n\\
  \text{When we double $n$ and $t$:}\\
  2n = \sqrt[\leftroot{-0}\uproot{3}3]{2t}\\
  2n = 1.25992104989t\\
  30 / 1.25992104989 \approx 24\\
\end{gather*}

\subsection{The master theorem}

\marginpar{See pages 268-270 in the course textbook for more information}

The master method is a way of solving divide and conquer recurrence equations
without having to explicitly use induction. It is used when an algorithm's
complexity is of the form:

\[
  T(n) = 
  \begin{cases}
    c               & \text{if $n \le d$}\\
    aT(n/b) + f(n)  & \text{if $n \geq d$}
  \end{cases}
\]

Where $f(n)$ is a function that is positive when $n \geq d$ and:

\begin{tabular}{>{$}l<{$}}
  d \geq 1\\
  a > 0\\
  b > 1\\
  c > 0\\
  d \in \mathbb{Z}\\
  a,b,c \in \mathbb{R}
\end{tabular}

Such a recurrence relation occurs whenever an algorithm uses a divide and
conquer approach. Such an algorithm will split the problem into $a$ subproblems,
each of size $n/b$ before recursively solving them and merging the result back
together. In this case, $f(n)$ is the time it takes to split the problem into
subproblems and merge them back together after the solving is done.

The master theorem is defined by three cases:

\begin{enumerate}
  \item If there is a small constant $\epsilon > 0$ such that $f(n)$ is
        $O(n^{log_{b}{a-\epsilon}})$, then $T(n)$ is $\Theta(n^{log_ba})$

        An example could be when the recurrence relation is
        \[
          T(n) = 4T(n/2) + n
        \]
        since $n^{log_{b}{a}} = n^{log_{2}{4}} = n^2$, therefore $T(n) =
        \Theta(n^{log_24}) = \Theta(n^2)$.
  \item % TODO:
        Work out what's going on...
\end{enumerate}

\section{Algorithmic correctness}

\section{Data structures}

\section{Basic algorithms}

\subsection{Sorting}

A sorting algorithm takes as an input, an array of keys, where there is a
\textit{total order} on the keys (each key can be compared to another), and
produces as an output, the array where the keys are ordered according to their
order.

A total order is a relation that is transitive ($a \leq b \wedge b \leq c \implies a
\leq c$), anti-symmetric ($a \leq b \wedge b \leq a \implies a = b$), and
unsuprisingly, total ($a \leq b \wedge b \leq a$).

If the ordering on the elements of the array isn't a total order then bad
things can happen when you try and use an algorithm to sort the data. For
example, if you were to try and sort an array of rocks, papers and scissors
according to the rules of the traditional game, then you wouldn't have a
transitive relation, and a sorting algorithm would probably loop infinately.

\subsubsection{Define sorting}

\subsubsection{Quicksort}

\subsubsection{Merge sort}

\subsubsection{Radix sort and Bucket sort}

\subsubsection{Heap sort}

\subsubsection{The lower bound of sorting}



\subsection{Searching}

\subsection{Tree and graph traversal}