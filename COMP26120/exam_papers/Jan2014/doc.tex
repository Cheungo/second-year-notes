\documentclass{article}

% Tell me about my LaTeX bad practices
\usepackage[l2tabu, orthodox]{nag}
% Make the margins wider
\usepackage[margin=1in]{geometry}
% For nice headers
\usepackage{fancyhdr}
% For line brakes in tables
\usepackage{tabularx}
% For the split environment
\usepackage{amsmath}
% For tabs in verbatim
\usepackage{moreverb}
% Means you don't have to put \\ to start a new line.
\usepackage[parfill]{parskip}
% For code listings
\usepackage{listings}
% For code listing colours
\usepackage{color}
% For images
\usepackage[pdftex]{graphicx}
% Make LaTeX pretty with better kerning etc
\usepackage{microtype}

\definecolor{mygreen}{rgb}{0,0.6,0}
\definecolor{mygray}{rgb}{0.5,0.5,0.5}
\definecolor{mymauve}{rgb}{0.58,0,0.82}

\lstset{ %
  backgroundcolor=\color{white},   % choose the background color; you must add \usepackage{color} or \usepackage{xcolor}
  basicstyle=\footnotesize,        % the size of the fonts that are used for the code
  breakatwhitespace=false,         % sets if automatic breaks should only happen at whitespace
  breaklines=true,                 % sets automatic line breaking
  captionpos=b,                    % sets the caption-position to bottom
  commentstyle=\color{mygreen},    % comment style
  deletekeywords={...},            % if you want to delete keywords from the given language
  escapeinside={\%*}{*)},          % if you want to add LaTeX within your code
  extendedchars=true,              % lets you use non-ASCII characters; for 8-bits encodings only, does not work with UTF-8
  frame=single,                    % adds a frame around the code
  keepspaces=true,                 % keeps spaces in text, useful for keeping indentation of code (possibly needs columns=flexible)
  keywordstyle=\color{blue},       % keyword style
  language=Java,                 % the language of the code
  morekeywords={*,...},            % if you want to add more keywords to the set
  numbers=left,                    % where to put the line-numbers; possible values are (none, left, right)
  numbersep=5pt,                   % how far the line-numbers are from the code
  numberstyle=\tiny\color{mygray}, % the style that is used for the line-numbers
  rulecolor=\color{black},         % if not set, the frame-color may be changed on line-breaks within not-black text (e.g. comments (green here))
  showspaces=false,                % show spaces everywhere adding particular underscores; it overrides 'showstringspaces'
  showstringspaces=false,          % underline spaces within strings only
  showtabs=false,                  % show tabs within strings adding particular underscores
  stepnumber=1,                    % the step between two line-numbers. If it's 1, each line will be numbered
  stringstyle=\color{mymauve},     % string literal style
  tabsize=2,                       % sets default tabsize to 2 spaces
  title=\lstname                   % show the filename of files included with \lstinputlisting; also try caption instead of title
}

\pagestyle{fancyplain}

\author{Todd Davies}
\title{COMP26120 - January 2014 - Answers}
\date{\today}

\begin{document}

\rhead{COMP26120 - January 2014 - Answers}
\lhead{\today}

\maketitle

\begin{center}
	\small Please don't assume these answers are right. This is me attempting a
	past paper for revision purposes; I could have got it all wrong ;)

  I chose to answer questions 1 \& 2.
\end{center}

\section*{Question 1}

\subsection*{Part a}

\lstinputlisting[language=Java, firstline=2, lastline=12, caption={
	Add all of the pairs of integers together in the list. Do this by looping
	through the list, and looping from 0$\rightarrow$i on each
	iteration.
}]{code/part1a.java}

Since as we're iterating through the list, we are describing an arithmetic series
(we do one additions, then two, then three etc), we can use the formula
$\frac{n(1+n)}{2}$ to find how many operations we're doing. This, in the Big-Oh
notation, equates to $O(n^2)$ since constant terms are eliminated. In most
cases, the result will be found significantly faster though.

\newpage

\subsection*{Part b}

\lstinputlisting[language=Java, firstline=2, lastline=11, caption={
	Add the first list to a BitSet (a very compact array of bits), then iterate
	through the second list and add items that are in the bit set to the output
	array.
}]{code/part2a.java}

The worst case runtime of this algorithm is $O(n + m)$ since it iterates through
both arrays once, and the loops aren't nested.

\newpage

\subsection*{Part c}

\lstinputlisting[language=Java, firstline=5, lastline=15, caption={
	Create a new array containing the first half, and a new array containing the
	second half, then replace them insert them into the original array.
}]{code/part3a.java}

The worst case runtime of this function is $O(n)$, since it only iterates
through the array once. The space complexity is $O(2n)$, since I couldn't figure
out how to do an in-place circular shift.

\section*{Question 2}

\subsection*{Part a}

\subsubsection*{i)}

Merge Sort

\subsubsection*{ii)}

Quick Sort

\subsubsection*{iii)}

Radix Sort

\subsection*{Part b}

\subsubsection*{i)}

It never ends.

\subsubsection*{ii)}

The average time that the smallest element will take to reach it's correct
position is $n \cdot n$, since we need one die roll to be 0, and the other to be
the current index of the smallest element, and there is a $\frac{1}{n}$ chance
of both occurring, so there is a $\frac{1}{n^2}$ chance of them occurring on the
same iteration.

In order for all of the $n$ items in the list to settle into their current
position in the list, we need the above to happen $n$ times, so we end up with
an average running time of $n \cdot n \cdot n = O(n^3)$.

\subsection*{iii)}

If every roll of the dice happened to be a valid swap, and the first roll was
swapping the smallest element with the element at position $0$, the second roll
was swapping the next smallest element with the one at position $1$, and so on,
then after $O(n)$ iterations, the list would be sorted.

\subsection*{Part c}

\subsubsection*{i)}

\begin{itemize}
  \item If $x > y$ and $y > z$ then $x > z$ (transitivity).
  \item If $x > y$ and $y > x$ if and only if $x = y$ (anti-symmetry).
\end{itemize}

\subsubsection*{ii)}

The notation for lower bound is big omega ($\Omega$).

Comparison based sorts must at have a runtime of at least $n \log n$, therefore
their asymptotic running time is $\Omega(n \log n)$.

\subsubsection*{iii)}

The reason that comparison based sorts must be $\Omega(n \log n)$ is that they
sorting algorithms must satisfy the transitivity condition.

Comparison sorts basically just traverse a decision tree of comparisons of
elements in the list. The tree can have at most $n!$ leaves (since that's how
many permutations of a list of length $n$ there are). An optimally balanced
decision tree with $n!$ leaves would be $\log_2(n!)$ levels deep. This means we
have to do at most $\log_2(n!)$ comparisons, and by Stirling's approximation:

\begin{align*}
  \Omega(\log_2(n!)) \approx \Omega(\log_2(n!))
\end{align*}

\end{document}