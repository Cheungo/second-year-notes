
% So the image fits the page
\documentclass{article}
\usepackage[paperwidth=\maxdimen,paperheight=\maxdimen]{geometry}
% Include tikz
\usepackage{tikz}
% For making the page fit the image (without standalone)
\usepackage[tightpage,active]{preview}
\PreviewEnvironment{tikzpicture}
\usetikzlibrary{shapes, patterns}
\usetikzlibrary{matrix}
% Define some lovely colours
\definecolor{light-gray}{gray}{0.95}

\def\leftbracket{[}
\def\rightbracket{]}

\tikzset{my arrow/.style={
  blue!60!black,
  -latex
  }
}


\begin{document}

\begin{center}
  \begin{tikzpicture}
    \matrix[matrix of math nodes, row sep=4mm] (M) {
      \leftbracket & 1 & 2 & 3 & 4 & 5 & 6 & 7 & 8 & \rightbracket \\
      \leftbracket & 1 & 2 & 3 & 4 & 5 & 6 & 7 & 8 & \phantom{9} & \phantom{9} & \phantom{9} & \phantom{9} & \phantom{9} & \phantom{9} & \phantom{9} & \phantom{9} \rightbracket \\
      \leftbracket & 1 & 2 & 3 & 4 & 5 & 6 & 7 & 8 & 9 & \phantom{9} & \phantom{9} & \phantom{9} & \phantom{9} & \phantom{9} & \phantom{9} & \phantom{9} \rightbracket \\
    };
    \node[font=\fontsize{10}{12}\sffamily, node distance = 4cm, right of=M] {\textsterling 2};

    \draw[my arrow] (M-1-2) to (M-2-2);
    \draw[my arrow] (M-1-3) to (M-2-3);
    \draw[my arrow] (M-1-4) to (M-2-4);
    \draw[my arrow] (M-1-5) to (M-2-5);
    \draw[my arrow] (M-1-6) to (M-2-6);
    \draw[my arrow] (M-1-7) to (M-2-7);
    \draw[my arrow] (M-1-8) to (M-2-8);
    \draw[my arrow] (M-1-9) to (M-2-9);
  \end{tikzpicture}
\end{center}

\end{document}