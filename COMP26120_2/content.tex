% Set the author and title of the compiled pdf
\hypersetup{
	pdftitle = {\Title},
	pdfauthor = {\Author}
}

\section{Trees}

\subsection{Motivation}

A tree is an abstract data type for heirarchical storage of information.

\subsection{Definition}

Each element in a tree has a parent element, and zero or more children elements.
The node at the top of the tree is called the root.

A tree is said to be \textit{ordered} if a linear ordering relation is defined
for the children of each node, that is to say that if we wanted to, we could
apply this relation to sort the children into an ordered list.

A binary tree is one where each node can have a maximum of two children. A
binary tree is \textit{proper} if each node as two (or zero) children.

The depth of a node is the number of ancestors of the node exclusing the node
itself.

% TODO: mention that recursive algorithms can be slooooowww for large trees
%       compared to an equivalent iterative algorithm. 

\subsubsection{Tree methods}

\begin{description}
	\item \texttt{isInternal}\\
		Complexity = O(1)
	\item \texttt{isExternal}
	\item \texttt{isRoot}\\
	\item \texttt{size}\\
	\item \texttt{elements}\\
	\item etc
\end{description}

\subsection{Tree traversal}

Pre-order, in-order and post-order traversal.
