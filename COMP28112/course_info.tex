\section*{Introduction}

Many of the most important and visible uses of computer technology rely on
distributed computing. Understanding distributed computing requires an
understanding of the problems and the challenges stemming from the coordinated
operation of different hardware and software. The course focuses on a set of
common techniques required to address the key challenges of distributed
computing.

\section*{Aims}

Many of the most important and visible uses of computer technology rely on
distributed computing. This course unit aims to build on the course unit in the
first year (COMP10052) which introduced students to the principles of
distributed computing, and it focuses on techniques and methods in sufficient
breadth and depth to provide a foundation for the exploration of specific topics
in more advanced course units. The course unit assumes that students have
already a solid understanding of the main principles of computing within a
single machine, have a rudimentary understanding of the issues related to
machine communication and networking, and have been introduced to the area of
distributed computing.

\begin{mymulticols}
  \begin{itemize}
    \item Revision of the characteristics of distributed systems. Challenges. Architectural models.
    \item Remote Invocation and Distributed Objects
    \item Java RMI, CORBA, Web Services.
    \item Message-Oriented middleware
    \item Synchronous vs asynchronous messaging. Point-to-point messaging. Publish-subscribe.
    \item Concurrency, co-ordination and distributed transactions
    \item Ordering of events. Two-phase commit protocol. Consensus.
    \item Caching and Replication
    \item Security
    \item Service-Oriented Architectures, REST and Web Services
  \end{itemize}
\end{mymulticols}

\section*{Contributing}

These notes are open-sourced on Github at
\url{https://github.com/Todd-Davies/second-year-notes}. Please feel free to
submit a pull request if you want to make any changes, or maybe open an issue
if you find an error! Feedback is very welcome; you can email me at
\href{mailto:todd434@gmail.com}{todd434@gmail.com}.

\section*{Warning!}

Since I write notes on the course before I start to look at past exam papers, I
don't always produce notes that are in line with what is examined. Though these
notes are still relevant to the course, I seem to have skimmed over topics that
are heavily examined and written detailed content on stuff that never comes up.
Sorry!
