\answer{What is a Java servlet?}{2}{2012.1.a}

A Java Servlet is a Java program with the capabilities of a server. It could
host web pages, provide an API endpoint or other services and is most commonly
used to serve content via the \texttt{HTTP} protocol.

\answer{What is the main assumption on which Cristian’s clock synchronisation
algorithm is based?}{2}{2012.1.b}

The time for a message to go from machine A to machine B is (roughly) the same
as the time it would take for a message to go from machine B to machine A. This
is most often accurate for routes with small round trip times.

\answer{Explain the difference between a name server and a directory
server.}{2}{2012.1.c}

A name server takes a name, matches it to an object and returns attributes about
the object. A directory server takes attributes, matches them to an object and
returns more attributes about the object.

\answer{Explain briefly what is meant by the term middleware.}{2}{2012.1.d}

Middleware is software that sits between a client application and the operating
system. It provides services to the client application that the operating system
does not, such as RPC stubs.

It can also provide an abstraction from the OS to mask the heterogeneity of
platforms used in distributed systems (some apps will run on Windows, others on
Linux, some on x64, some on ARM architectures etc).

\answer{Why is it practically impossible to achieve strict consistency in a
distributed  system?}{2}{2012.1.e}

Strict consistency is when any read to a shared data item returns the most
recent write operation on that data item. This means there must be an absolute
time ordering of all accesses. Unfortunately, since (as we know from the eight
fallacies of Distributed Computing), the latency of any network is not zero and
messages are not reliable. This means that any message we send to update other
machines about a change of state may not be sent. Since there is no way of
getting an absolute global clock or getting any global state, then we cannot
achieve real time memory consistency across all nodes, and therefore cannot
achieve strict consistency in the system.

\answer{Traditional RPC mechanisms cannot handle pointers. What is the
problem?}{2}{2012.1.f}

In order to handle pointers with RPC, you must serialise the datastructure into
a message so that it can be constructed from the same message at the other
machine. When the reply is received back at the sender, it can be deserialised
again and the values in the datastructure that were changed on the remote
machine can be updated in local memory. Traditional RPC mechanisms didn't have
this facility.
